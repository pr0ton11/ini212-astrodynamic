\chapter{Technische Dokumentation}

\section{Applikationsdokumentation}

\subsection{Maven build}

\subsection{Diagramme}

\section{Themenumsetzung}

\subsection{Unit Tests}

Unsere Unit Tests befinden sich im Ordner \textit{"src/test/java/ch/hftm/astrodynamic"} und sind nach Anwendungskomponenten gegliedert:

\begin{itemize}
	\item DAO (Data Access Object); Testet den Zugriff und die Fuktionalität des zentrale MissionRepositorys (in welchem alle Missionen und deren Kindobjekte gespeichert sind)
	\item Model; Testet die pedefinierten Modelle (z.B Mond, Erde, Sonne) welche mit der Software mitgeliefert werden
	\item Planetoid; Testet die Funktionalität der Planetoiden (also planetenähnlichen Objekten)
	\item Quad; Testet die Implementation und Abstraktion unseres Quad Datentyps, welcher für alle Berechnungen innerhalb des Projekts verwendet wird
	\item Scalar; Testet die Richtigkeit von Scalar basierten Verrechnungen
	\item SimulationTest; Testet die Simulation der einzelnen Objekte im Sonnensystem
	\item Unit; Testet die verschiedenen Scalar Physik-Einheiten und deren Verrechnungen
	\item Vector; Testet die Vektoren und deren Verrechnungen
\end{itemize}

Wir haben uns primär dafür entschieden, die Business-Logik (primär die Simulation) zu testen, da die Erfolg unseres Projekts primär von der Korrektheit der Simulation abhängig ist.

\subparagraph{Unit Tests ausführen}

Zunächst müssen wir uns ins \textit{astrodynamics} Verzeichnis bewegen in welchem ebenfalls die Datei \textit{pom.xml} gespeichert ist.
Anschliessend können wir die Unit-Tests mit einem einfachen \textit{"mvn clean test"} ausführen.

\subsection{Enumeration}

Unser wichtigster Einsatz von Enum (Enumeration) findet in der Klasse \textit{Unit} statt. Diese Klasse wird dazu genutzt, SI Einheiten, welche in unserem Projekt verwendet werden, darzustellen:

\begin{lstlisting}
	// Contains all physical units as enum
	public enum Unit {
		TIME, // Seconds (s)
		LENGTH, // Meters (m)
		MASS, // Kilogram (kg)
		CURRENT, // Ampere (A)
		TEMPERATURE, // Kelvin (K)
		MOLECULES, // Mol (mol)
		LUNINOSITY, // Candela (cd)
		// Extensions (Implicit)
		VOLUME,  // m ^ 3
		AREA, // m ^ 2
		FORCE, // N
		ACCELERATION, // m/s ^ 2
		VELOCITY, // m/s
		ANGLE, // radian
		ANGULAR_VELOCITY, // radian/second
		ANGULAR_ACCELERATION, // radian/second^2
		// kg ^ 2 imaginary unit for gravitational calculations
		CUBIC_MASS, 
		// kg ^ 2 / m ^ 2 imaginary unit for gravitational calculations
		M2_DIV_L2,
		// N * m ^ 2 * kg ^ 2 gravitational constant
		//  (s ^ -2 * m ^ 3 * kg ^ - 1) 
		F_L2_Mn2,
		// Unitless for scalars without unit
		UNITLESS
	}
\end{lstlisting}

Wir nutzen dieses Enum in allen Skalaren und in diversen Vektoren, um die Einheiten der Werte darzustellen. z.B nutzt die Klasse \textit{LengthScalar}, welche eine Länge speichert, den Wert \textit{LENGTH}.

In den Kommentaren der Klasse werden die genauen Einheiten explizit definiert. Dies hilft bei den Umrechnungen der Werte z.B zwischen verschiedenen Zeiteinheiten (z.B zwischen Stunden, Tagen oder Wochen).

\section{Vererbung}

Wir nutzen Vererbung in allen unseren Klassen. Spezifisch hervorheben möchten wir in diesem Fall die \textit{Scalar} Klassen im Package \textit{ch.hftm.astrodynamic.scalar}, welche allesamt von der Klasse \textit{BaseScalar} im Package \textit{ch.hftm.astrodynamic.utils} erben.

Die einzelnen \textit{Scalar} Klassen stellen jeweils einen spezialisierten Scalar im Projekt dar. z.B werden Längen mittels dem \textit{LengthScalar} dargestellt.
Die Einheiten der jeweiligen Klassen werden in der Kind-Klasse innerhalb des Konstruktors statisch gesetzt und dem Eltern-Konstruktor des \textit{BaseScalar} übergeben und gespeichert.

\section{Casting}

Casting wird in einzelnen Fällen innerhalb der Controller verwendet, um z.B einen Double zu einem Integer zu Runden.

\section{Interfaces}

Interfaces werden in unserem Projekt häufig verwendet. Vorwiegend haben wir mittels Interfaces generalisierte Klassen (und deren Vererbungen) implemtiert.

Als Beispiel dient hier das Interface \textit{Scalar} aus dem Package \textit{ch.hftm.astrodynamics.utils}, welches z.B für generalisierte Operationen innerhalb aller Klassen verwendet wird. Mit diesem Interface können wir sicherstellen, dass alle \textit{Scalar} Klassen verglichen oder berechnet werden können.

Ebenfalls ist in diesem Interface eine Implementation für die Funktion \textit{toFittedString()} bereits vorgegeben (und muss aus diesem Grund nicht erneut implementiert werden).

